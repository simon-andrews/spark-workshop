\section{Building a calculator}

\begin{frame}{Your mission}
\begin{itemize}
    \item ``Hello, World!'' is boring. Let's build something that can interact with the user.
    \item In this workshop, we'll be building a four function ($+$, $-$, $\times$, $\div$) calculator.
    \item Our interface will work like this: for every operation, \url{http://localhost:4567/n/op/m} will display the result of applying the operation \textit{op} to the operands \textit{n} and \textit{m}.
    \item For example, \url{http://localhost:4567/4/plus/2} will display 6.
\end{itemize}
\end{frame}

\begin{frame}{GET requests with parameters}
\begin{itemize}
    \item Programming \texttt{/0/plus/0}, \texttt{/0/plus/1}, \texttt{/0/plus/2}, etc. would take a pretty long time.
    \item Luckily, we can tell Spark to look for patterns in the requested resources and interpret them.
    \item We can make a route that looks like ``$n$/$op$/$m$'', then in the function for the route we can work with those values.
\end{itemize}
\end{frame}

\begin{frame}[fragile]{Introducing the calculator!}
Update App.java:
\begin{minted}{java}
package org.simonandrews.sparktutorial;

import static spark.Spark.*;

public class App {
  public static void main(String[] args) {
    get("/:n/:op/:m", (req, res) -> "TODO");
  }
}
\end{minted}
Try compiling and running this program and accessing some URLs with your web browser. What works? What doesn't?
\end{frame}

\begin{frame}{URL parameters with Spark}
\begin{itemize}
    \item ``:n'', ``:op'', and ``:m'' are allowed to be any string.
    \item Our next step will be to make sure that they're what we expect them to be.
    \item This is super important:
    \begin{itemize}
        \item Makes sure our application behaves the way we expect it to.
        \item For a ``real world'' application, this might be important for security as well.
    \end{itemize}
\end{itemize}
\begin{center}
\includegraphics[width=15em]{resources/xkcd_327.png} \\
\textit{Comic by Randall Munroe of xkcd: \url{https://www.xkcd.com/327/}}
\end{center}
\end{frame}

\begin{frame}[fragile]{Playing with parameters}
We can get the value of a URL parameter with the request object's \texttt{param} method:
\begin{minted}{java}
package org.simonandrews.sparktutorial;

import static spark.Spark.*;

public class App {
  public static void main(String[] args) {
    get("/:n/:op/:m", (req, res) -> req.params(":n"));
  }
}
\end{minted}
Now every request will return whatever $n$ was. For example, \url{http://localhost:4567/1/plus/3} will display 1.
\end{frame}

\begin{frame}[fragile]{Doing the computation}
With that in our toolkit, let's build out our function a bit:
\begin{minted}{java}
//omitted...
get("/:n/:op/:m", (req, res) -> {
  double n = Double.parseDouble(req.params(":n"));
  double m = Double.parseDouble(req.params(":m"));
  switch (req.params(":op")) {
    case "plus":  return String.valueOf(n + m);
    case "minus": return String.valueOf(n - m);
    case "times": return String.valueOf(n * m);
    case "div":   return String.valueOf(n / m);
    default:      return "oops!";
  }
});
//omitted...
\end{minted}
    
\end{frame}

\begin{frame}{Try it out!}
\begin{itemize}
    \item As usual, \texttt{./gradlew build} and \texttt{java -jar build/libs/spark-tutorial.jar}!
    \item Try your operations out. For example, what's displayed when you visit \url{http://localhost:4567/1/minus/0.5}?
\end{itemize}
\end{frame}
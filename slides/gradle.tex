\section{Gradle crash course}

\begin{frame}{But before we start with that\dots}
\begin{itemize}
    \item We're in the hands-on portion now!
    \item Go to \textit{some URL} and download and extract the ``stage 0'' ZIP file.
    \item Next, fire up a terminal and \texttt{cd} over to the folder you just extracted.
\end{itemize}    
\end{frame}

\begin{frame}{What is Gradle?}
%\begin{center}
%    \includesvg[width=10em]{resources/maven_logo.svg}
%\end{center}
\begin{itemize}
    \item Gradle is a \textbf{build automation} tool popular with Java developers.
    \item Gradle's main jobs are to
    \begin{itemize}
        \item manage your project's \textbf{dependencies}, and
        \item control building your project's \textbf{artifacts}.
    \end{itemize}
    \item Learn more: \url{https://gradle.org}
\end{itemize}    
\end{frame}

\begin{frame}{Why should you care?}
\begin{itemize}
    \item Gradle can be used by anyone in pretty much any workflow.
%    \begin{itemize}
%        \item Not everyone uses Eclipse and can just push the little green ``run'' button.
%        \item Sometimes non-humans will want to build your program (mostly continuous integration servers)
%    \end{itemize}
    \item Gradle helps you share your code and use code other people shared with you.
%    \begin{itemize}
%        \item Gradle encourages you to structure your code in a standardized way, and helps you produce binaries you can easily share with others.
%        \item Gradle helps you download other people's code from the Internet and use it in your own projects.
%    \end{itemize}
\end{itemize}
\end{frame}

\begin{frame}[fragile]{How to control Gradle: build.gradle}
Here's what a minimal build.gradle looks like for a Java project.
\begin{minted}{groovy}
plugins {
  id 'java'
}
\end{minted}
It can be this small because Gradle makes a lot assumptions about your project's structure.
\end{frame}

\begin{frame}{Organizing your sources for Gradle}
\begin{itemize}
    \item src/ -- Source code
    \begin{itemize}
        \item main/ -- Application code
        \begin{itemize}
            \item java/ -- Java source code
            \item resources/ -- Other files (configuration, data, etc.)
        \end{itemize}
        \item test/ -- Unit tests
    \end{itemize}
    \item build.gradle -- Instructions for building your project
    \item settings.gradle -- Extra configuration stuff for Gradle
\end{itemize}
You can see some of these in the project from the stage 0 ZIP.
\end{frame}

\begin{frame}[fragile]{Adding dependencies to build.gradle}
\begin{minted}{groovy}
plugins {
  id 'java'
}

repositories {
  jcenter()
}

dependencies {
  implementation 'com.sparkjava:spark-core:2.9.1'
}
\end{minted}
\end{frame}

\begin{frame}[fragile]{About JARs}
A \textbf{JAR} file (\textbf{J}ava \textbf{AR}chive) is a container that you can put compiled Java code (+ other stuff) into one neat runnable package.
\begin{minted}{groovy}
jar {
  manifest {
    attributes(
      'Main-Class':
        'org.simonandrews.sparktutorial.App'
    )
  }
  from {
    configurations.runtimeClasspath.collect {
        it.isDirectory() ? it : zipTree(it)
    }
  }
}
\end{minted}
\end{frame}

\begin{frame}{Actually using Gradle}
\begin{itemize}
    \item From build.gradle, Gradle defines \textbf{targets}, actions you can run.
    \item You can run these from the command line: ./gradlew \textit{sometarget}.
    \item Try it:
    \begin{itemize}
        \item Build: \texttt{./gradlew build}
        \item Run: \texttt{java -jar build/libs/spark-tutorial.jar}
    \end{itemize}
    \item Using Eclipse? Run \texttt{./gradlew eclipse} to turn your directory into an Eclipse project, then import it!
\end{itemize}
\end{frame}
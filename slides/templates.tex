\section{Templates}

\begin{frame}{The problem}
\begin{itemize}
    \item Baking HTML into Java code is terrible.
    \item You need to escape all your quotes.
    \item You need to keep track of all the strings you're concatenating.
    \item You editor doesn't understand you're writing HTML, so you won't get autocompletion or syntax highlighting.
    \item What should your indenting rules even be?
    \item What about CSS? What about JavaScript?
    \item What about long HTML documents?
\end{itemize}
\end{frame}

\begin{frame}{The solution}
Separate your front-end from your back-end!
\begin{itemize}
    \item We will do this with \textbf{templates}, files that have spaces you can fill in dynamically.
    \item Basically, we can make a template ``\$n \$op \$m = \$result'' and then render it once we know the values for all those variables.
    \item We can do this in a separate file, and include that file in our JAR along with all our .classes. (Remember the resources directory from earlier?)
\end{itemize}
\end{frame}

\begin{frame}{Introducing Velocity!}

\begin{center}
\includesvg[width=20em]{resources/velocity_logo.svg}
\end{center}
\begin{itemize}
    \item Velocity is a library for rendering templates. It has its own template language designed specifically to work nice with Java programs.
    \item We won't use them today, but you can use for-loops and if-statements in it too. Velocity is very flexible!
\end{itemize}
\end{frame}

\begin{frame}[fragile]{Adding the dependency}

In build.gradle, change the \texttt{dependencies} section to
\begin{minted}[fontsize=\footnotesize]{groovy}
dependencies {
  implementation 'com.sparkjava:spark-core:2.9.1'
  implementation 'com.sparkjava:spark-template-velocity:2.7.1'
}
\end{minted}
This adds a dependency on an adapter for Velocity that makes it work nicely with Spark.
\end{frame}

\begin{frame}{Building our first template}
\begin{itemize}
    \item Files in the ``src/main/resources'' will be included in your JAR without any modification.
    \item Your Java code can then access those files.
    \item This is where we'll be putting our templates.
\end{itemize}
\end{frame}

\begin{frame}[fragile]{Adding our templates}
Put the following in src/main/resources/form.vtl:
\begin{minted}{html}
<form action="/" method="POST">
  n: <input type="number" step="any" name="n" /> <br />
  op:
  <select name="op">
    <option value="plus">+</option>
    <option value="minus">-</option>
    <option value="times">*</option>
    <option value="div">/</option>
  </select>
  <br />
  m: <input type="number" step="any" name="m" /> <br />
  <input type="submit" value="Go!" />
</form>
\end{minted}
\end{frame}

\begin{frame}[fragile]{Adding our templates}
Put the following in src/main/resources/result.vtl:
\begin{minted}{html}
<p>The result of $n $op $m is $result.</p>
\end{minted}
\end{frame}

\begin{frame}[fragile]{Wiring up Spark}
In App.java:
\begin{minted}{java}
import java.util.HashMap;
import java.util.Map;
import spark.ModelAndView;
import spark.template.velocity.VelocityTemplateEngine;
// omitted...
get("/", (req, res) -> {
    Map<String, Object> emptyModel = new HashMap<>();
    return new ModelAndView(emptyModel, "form.vtl");
}, new VelocityTemplateEngine());
// TODO: post
// omitted...
\end{minted}
\end{frame}

\begin{frame}[fragile]{The POST part}
\begin{minted}[fontsize=\footnotesize]{java}
post("/", (req, res) -> {
  String op = req.queryParams("op");
  Double n = Double.parseDouble(req.queryParams("n"));
  Double m = Double.parseDouble(req.queryParams("m"));
  Double r;
  switch (op) {
    case "plus":  r = n + m; break;
    case "minus": r = n - m; break;
    case "times": r = n * m; break;
    case "div":   r = n / m; break;
    default:      throw new Exception("oops");
  }
  Map<String, Object> model = new HashMap<>();
  model.put("n", n); model.put("m", m);
  model.put("op", op); model.put("result", r);
  return new ModelAndView(model, "result.vtl");
}, new VelocityTemplateEngine());
\end{minted}
\end{frame}

\begin{frame}{Voilà!}
It works (hopefully)! Go you!
\end{frame}
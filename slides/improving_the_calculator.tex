\section{Improving the calculator}

\begin{frame}{Problems with the old system}
\begin{itemize}
    \item It's ugly.
    \item Terrible user experience.
    \item No one uses the Internet this way.
\end{itemize}
(It might be nice as an API though!)
\end{frame}

\begin{frame}{What's the real solution?}
\begin{itemize}
    \item Wouldn't it be better if the user could just fill out a form, click submit, and get their result? There would be no need to fiddle with URLs by hand.
    \item Let's do exactly that!
    \item HTML has some support for forms built in with the \texttt{<form>} tags.
\end{itemize}
\end{frame}

\begin{frame}{Remember POST requests?}
\begin{itemize}
    \item We will replace our ``/:n/:op/:m'' endpoint with one single endpoint ``/''. This is your website's \textbf{root}, which \textit{basically} means it's what we get when we visit \url{http://localhost:4567} without specifying a resource.
    \item / can be acted upon using two methods:
    \begin{itemize}
        \item When the client GETs /, they will receive a form asking them for a calculation.
        \item When the client POSTs to / with some numbers and an operation, the server will do the requested computation and the client will receive the result.
    \end{itemize}
\end{itemize}
\end{frame}

\begin{frame}[fragile]{POSTing from a form}
Here's what the HTML code for our form will look like:
\begin{minted}{html}
<form action="/" method="POST">
  n: <input type="number" step="any" name="n" /> <br />
  op:
  <select name="op">
    <option value="plus">+</option>
    <option value="minus">-</option>
    <option value="times">*</option>
    <option value="div">/</option>
  </select>
  <br />
  m: <input type="number" step="any" name="m" /> <br />
  <input type="submit" value="Go!" />
</form>
\end{minted}
\end{frame}

\begin{frame}{What happens on the client side?}
When the user clicks the ``Go!'' button in their web browser, the browser will make a POST request to / with string $n$, $op$, and $m$ as parameters.

It is important to remember that \textit{these strings could be anything}, even though we try to constrain the user's input using HTML.
\end{frame}

\begin{frame}[fragile]{What happens on the server side?}
We need to tell Spark how to handle these new request types. Let's try a simple example first. Replace App.java with:
\begin{minted}{java}
package org.simonandrews.sparktutorial;

import static spark.Spark.*;

public class App {
  public static void main(String[] args) {
    get("/", (req, res) ->
        "You did a GET!" +
        "<form action=\"/\" method=\"POST\">" +
        "<input type=\"submit\" value=\"Now POST!\"/>" +
        "</form>");
    post("/", (req, res) -> "You did a POST!");
  }
}
\end{minted}
\end{frame}

\begin{frame}{What just happened?}
\begin{itemize}
    \item We just defined two different methods for the same resource.
    \item When you GET /, which is what your browser does by default, you see the message ``You did a GET!'' and a button.
    \item When you POST /, which is what the form code tells your browser to do, you see the message ``You did a POST!''.
\end{itemize}
\end{frame}

\begin{frame}[fragile]{Now let's calculate!}
\begin{minted}{java}
// omitted...
String form = "<form action=\"/\" method=\"POST\">" +
              // omitted
              "</form>";
get("/", (req, res) -> form);
post("/", (req, res) -> {
  double n = Double.parseDouble(req.queryParams("n"));
  double m = Double.parseDouble(req.queryParams("m"));
  switch (req.queryParams("op")) {
    case "plus":  return String.valueOf(n + m);
    case "minus": return String.valueOf(n - m);
    case "times": return String.valueOf(n * m);
    case "div":   return String.valueOf(n / m);
    default:      return "oops!";
  }
});
// omitted...
\end{minted}
\end{frame}

\begin{frame}{Try it out!}
\begin{itemize}
    \item As usual, \texttt{./gradlew build} and \texttt{java -jar build/libs/spark-tutorial.jar}.
    \item Open \url{http://localhost:4567} in your browser.
    \item Fill out the form and click submit.
    \item Get your answer! Hooray, an interactive site!
\end{itemize}
\end{frame}
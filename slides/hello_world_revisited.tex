\section{``Hello, World!'' revisited}

\begin{frame}[fragile]{``Hello, World!'' again}
\begin{minted}{java}
package org.simonandrews.sparktutorial;

import static spark.Spark.*;

public class App {
    public static void main(String[] args) {
        get("/hello", (req, res) -> "Hello World");
    }
}
\end{minted}
\pause
Replace the code in \texttt{src/main/org/simonandrews/ sparktutorial/App.java} with this, then \texttt{./gradlew build}!
\end{frame}

\begin{frame}{Let's run it}
\begin{itemize}
    \item \texttt{java -jar build/libs/spark-tutorial.jar}
    \item If all you get is some complaining about ``SLF4J'' you did it right!
    \item Point your web browser to \url{http://localhost:4567/hello}. You should see a ``Hello, World!'' message.
    \item Stop the program by doing ``Ctrl-C'' in your terminal.
\end{itemize}
\end{frame}

\begin{frame}{What just happened?}
\begin{enumerate}
    \item When you ran the application with \texttt{java}, you started an HTTP server on your computer.
    \item The web server listen for connections on \texttt{port} 4567.
    \item When you connect to \url{http://localhost:4567/hello}, your web browser makes a GET request for the resource /hello on the server.
    \item The server matches your request to a function that generates a string, then sends that string back to your browser.
\end{enumerate}
\end{frame}